

\documentclass[10pt,a4paper]{moderncv}

% moderncv themes
\moderncvtheme[orange]{classic}                 % optional argument are 'blue' (default), 'orange', 'red', 'green', 'grey' and 'roman' (for roman fonts, instead of sans serif fonts)
%\moderncvtheme[green]{casual}                % idem, 

% character encoding
\usepackage[utf8]{inputenc}                   % replace by the encoding you are using

% adjust the page margins
%\usepackage[scale=0.80]{geometry}
\usepackage[left=1.75cm,top=1.65cm,right=2.28cm,bottom=2cm]{geometry} 
%\setlength{\hintscolumnwidth}{3cm}						% if you want to change the width of the column with the dates
%\AtBeginDocument{\setlength{\maketitlenamewidth}{6cm}}  % only for the classic theme, if you want to change the width of your name placeholder (to leave more space for your address details
\AtBeginDocument{\recomputelengths}   


%%%%%%%%%%%%%%%%%%%%%%%%%%%%%%%%%%%%%%%%%%%%%%%%%%%%%%%%%%%%%%%%%%%%%
% \usepackage{graphicx,amssymb,amstext,url}
% 
% %\usepackage{hyperref}
% \hypersetup{
%   pdfauthor={Rob Ward},
%   pdftitle={CV for Rob Ward},
%   pdfsubject={Detailed CV for Rob},
%   urlcolor=blue,
% }
% 
% 
% %\usepackage[toc]{glossaries}
% \usepackage[utf8]{inputenc}
% %\usepackage[spanish]{babel}
% \usepackage[cmex10]{amsmath}
% \usepackage{cite}
% \usepackage{array}
% \usepackage{mdwmath}
% \usepackage{mdwlist}
% \usepackage[center]{caption}
% % \usepackage[caption=false,font=footnotesize]{subfig}
% % \usepackage{stfloats}
% \usepackage{paralist}   %listas inline
% 
% 
% % \usepackage{wrapfig}
% \usepackage{booktabs}
% \usepackage{multirow}
% % \usepackage[normalem]{ulem}
% % \usepackage{soul}
% % \usepackage{longtable}
% \usepackage{epigraph}
% % \usepackage[table]{xcolor}
% % \usepackage[clockwise,figuresright]{rotating}
% %\usepackage[Bjornstrup]{fncychap}
% %\usepackage{indentfirst}
% \usepackage{lipsum}
% \usepackage{lastpage} % total page count
% \usepackage{setspace}
% \usepackage{textcomp}
% 
% \usepackage{totcount}
% 
 \usepackage{marvosym}
\usepackage{setspace}
 \usepackage{multicol} 
% %%%%%%%%%%%%%%%%%%%%%%%%%%%%%%%%%%%%%%%%%%%%
% 
% \usepackage{framed, color}
% 
% 







                  % required when changes are made to page layout lengths

% personal data
\firstname{Sara}
\familyname{Rodriguez Lopez}
%\title{\Large{Máster de Ingeniería Biomédica}}         
%\title{\Large{Ingeniera de Telecomunicación}}               % optional, remove the line if not wanted
%\address{DISAM-ETSII-UPM\\C/Jose Gutierrez Abascal,2}{28006 Madrid (SPAIN)}    % optional, remove the line if not wanted
%\address{}{Madrid}
%\mobile{+34 628436393}                    % optional, remove the line if not wanted
%\phone{+34 915514915}                      % optional, remove the line if not wanted
%\fax{fax (optional)}                          % optional, remove the line if not wanted
%\email{gonzalo.rcanosa@upm.es} 
%\email{sara@sararl.com}                       % optional, remove the line if not wanted
%\social[linkedin]{\url 
%\extrainfo{\linkedinsocialsymbol \href{http://es.linkedin.com/in/sararodriguezlopez}{sararodriguezlopez} \\ Date of birth: 17.12.1989} % optional, remove the line if not wanted
%\photo[50pt]{sara_elegante3.jpg}                         % '64pt' is the height the picture must be resized to and 'picture' is the name of the picture file; optional, remove the line if not wanted

%\quote{Curriculum Vit\ae}                 % optional, remove the line if not wanted
%\quote{ MSc in  Biomedical Engineering and MSc in Telecommunications Engineering, passionate about medical imaging. I like my work has an impact on improving people's health. 2 years working experience in international projects collaborating side by side with doctors and engineers.}% Elevada capacidad para manejar grandes bases de datos de información.}




%. Tras la formación recibida en la intensificación de Biongeniería realizada durante el último año de carrera en la ETSIT  y durante el año de colaboración en el grupo de investigación Biomedical Images Technologies de la ETSIT, busca aumentar los conocimientos en el campo de la bioingeniería. }
%\nopagenumbers{}                             % uncomment to suppress automatic page numbering for CVs longer than one page

%I build tools that help people improve their lives.

%----------------------------------------------------------------------------------
%            content
%----------------------------------------------------------------------------------


\newcommand{\SARAMOUSE}{{\Large \ComputerMouse \space}}
\newcommand{\SARAMOUSEnospace}{{\Large \ComputerMouse}}



\begin{document}


%\maketitle
\begin{minipage}{0.49\textwidth}
 
\begin{minipage}{0.33\textwidth}
{%
\setlength{\fboxsep}{2pt}%
\setlength{\fboxrule}{1pt}%
\fcolorbox{color1}{white}{\includegraphics[width=64pt]{sara_elegante3.jpg}}
}%
\end{minipage}
\begin{minipage}{0.68\textwidth}
{\Large \textbf{Sara Rodríguez López}} \vspace{5pt} \\
\textcolor{gray}
{ 
 \addresssymbol Madrid \\
 \mobilesymbol +34 628436393 \\
 \emailsymbol \href{mailto:sara@sararl.com}{sara@sararl.com}  \\
 \linkedinsocialsymbol \href{http://es.linkedin.com/in/sararodriguezlopez}{sararodriguezlopez} \\ 
 Date of birth: 17.12.1989
 }

\end{minipage}
\end{minipage}
\hfill
\begin{minipage}{0.49\textwidth}
%\onehalfspacing
\centering
\textcolor{orange}{MSc in  Biomedical Engineering and MSc in Telecommunications Engineering. \\Two years working experience in international projects collaborating with doctors and engineers. \vspace{8pt} \\ \textit{``I am passionate about medical imaging and I like that my work has an impact on improving people's health.''}}
\end{minipage}
% 


\section{Education}
\cventry{2013-2014}{MSc in Biomedical Engineering}{School of Telecommunications Engineering (ETSIT)}{Technical University of Madrid (UPM)}{}{Medical Imaging and Telemedicine specialization. Average mark: 8.89
}
\cventry{2007-2013}{MSc in Telecommunications Engineering}{Accreditation ABET/EAC}{}{ETSIT-UPM}{Bioengineering specialization. Average mark: 7.33}
%\cventry{2005-2007}{Bachillerato Científico-Tecnológico en CES Valdecás}{}{Madrid}{}{Calificación Prueba Acceso a la Universidad: 8.85}% arguments 3 to 6 are optional

%{Calificación: Matrícula de Honor}
%\cventry{Junio 2007}{Prueba Acceso a la Universidad}{}{Madrid}{}{Calificación: 8.85}
\section{Professional Experience}

%\subsection{Trabajo}
%\cventry{Oct.2013-Now}{Biomedical Image RD Engineer}{Contrato laboral}{Grupo de investigación Biomedical Image Technologies (BIT)}{ETSIT-UPM}{\textbf{- Funciones} \newline Realización de procesado y análisis de imagen en diversos proyectos. Utilización de Chest Imaging Platform  para la segmentación de vía aérea y vasos pulmonares en imágenes de tomografía computarizada (TC), y el software Weka para tareas de clasificación. También se ha trabajado con Matlab, Bash y Python. \vspace{0.2cm} \newline \textbf{-Principal proyecto:} \textit{Computer Aided Detection of Pulmonary Embolism (CAD-PE)}. \newline En colaboración con UPM, Massachusetts Institute of Technology (MIT), Brigham and Women's Hospital (BWH) de Boston y la Unidad Central de Radiodiagnóstico (UCR) de Madrid en el contexto del consorcio \href{http://mvisionconsortium.mit.edu/}{\textit{M+Visión}}. El objetivo del proyecto ha sido el desarrollo de un sistema de detección automática de tromboembolismos pulmonares sobre imágenes de angiografía por TC. \vspace{0.2cm} \newline \textbf{- Logros} \newline Trabajé con un equipo multidisciplinar de ingenieros y médicos de la UCR y el BWH, pudiendo en todo momento adaptar el trabajo a las necesidades reales finales. Mejoré notablemente el sistema de detección del proyecto \textit{CAD-PE}, disminuyendo los falsos positivos en un 70\% y obteniendo unos resultados que serán próximamente publicados. Previsiblemente mi trabajo será liberado e incorporado a Chest Imaging Platform, librería de herramientas para el análisis y procesado de imágenes torácicas.}


 \cventry{Oct.2013 Present}{Biomedical Image R\&D Engineer}{}{Biomedical Image Technologies (BIT) Research Group}{ETSIT-UPM}{\textbf{- Responsibilities} \newline Approach of processing and analysis of medical images in a few projects. Use of Chest Imaging Platform to segment airways and pulmonary vessels in Computed Tomography (CT) images, and Weka software to classification tasks. Matlab, Bash and Python have been also employed. \vspace{0.2cm}  \newline \textbf{- Major Project:} \textit{Computer Aided Detection of Pulmonary Embolism (CAD-PE)}. \newline In collaboration with \textit{Massachusetts Institute of Technology (MIT)}, \textit{ Brigham and Women's Hospital (BWH)} from Boston and \textit{UPM} and \textit{Unidad Central de Radiodiagnóstico} (UCR) from Madrid in the context of  \href{http://mvisionconsortium.mit.edu/}{\textit{M+Visión} consortium}. The goal of this project has been the development of a computer aided detection (CAD) system to automatically identify clots from CTPA images. \vspace{0.2cm} \newline \textbf{- Achievement Highlights} \newline 
 \begin{itemize} \item[$*$] I worked side by side with engineers and doctors from UCR, BWH and Hospital Gregorio Marañón, adjusting my work to final medical necessities. \item[$*$] I improved considerably the CAD system from project \textit{CAD-PE}, reducing false positives in a 70\% and getting a results that will be publish soon. \item[$*$] In April 2015, I was in New York presenting a poster about my work in automatic organs detection in \href{http://biomedicalimaging.org/2015/}{\texttit{ISBI}} conference, one of the most important conferences in biomedical imaging field. \end{itemize}}
                                                                                                                                                                                                                                                                                                                                                                                                                                                                                                                                                                                                                                                                                                                                                                                                                                                                                                                                                                                                                                                                                                                                                                                                                                                                                                                                                                                                                                                                                                                                                                                                                                                                         
% 



% ---VERSION 2
% Realización de procesado y análisis de imagen en diversos proyectos. Especial dedicación al proyecto \textit{“Computer Aided Detection of Pulmonary Embolism”}, utilizando la librería Chest Imaging Platform  para la segmentación de vía aérea y vasos pulmonares, y el software Weka para tareas de clasificación. También se ha trabajado con Matlab, Bash y Python. 
% ----ANTES
% Principalmente el trabajo se ha centrado en el desarrollo y optimización de un sistema para la detección automática de émbolos en imágenes de tomografía computarizada (TC), utilizando la librería Chest Imaging Platform  para la segmentación de vía aérea y vasos pulmonares, y el software Weka para tareas de clasificación. También se ha trabajado con Matlab, Bash y Python.


%Brigham and Women's Hospital (BWH) Applied Chest Imaging Lab's (ACIL) }
%\subsection{Estancia de investigación}
% \cventry{Oct.2013}{Brigham and Women's Hospital (BWH) - Massachusetts Institute of Technology (MIT) }{Boston,EEUU}{}{}{Con motivo del trabajo desarrollado durante el Proyecto Final de Carrera en colaboración con el MIT y el BWH se realizó una visita a dichas instalaciones para conocer al equipo con el que se había estado colaborando, finalizar los algoritmos y realizar la instalación y pruebas de la herramienta en el hospital. }
%\subsection{Becas}
%\cventry{}{}{}{}{}{}
\vspace{0.2cm}
%\cventry{Sept.2012-Jul.2013}{Ayudante de investigación}{Beca de colaboración del MECD otorgada por buen rendimiento académico}{Grupo de investigación BIT}{ETSIT-UPM}{\textbf{- Funciones} \newline Utilización de herramientas de visualización y análisis de imágenes médicas como Amide, ITKsnap y Amira. Segmentación de ventrículos de corazón humanos en imágenes de angiografía por TC y de cerdos en imágenes de resonancia magnética (RM) usando técnicas de level set implementadas en Matlab. Trabajo con algoritmos de detección basados en técnicas de aprendizaje máquina. \vspace{0.2cm} \newline \textbf{-Principal proyecto:} \textit{Right ventricular to left ventricular diameter ratio (RV/LV).} \newline En colaboración con UPM, MIT y BWH en el contexto del consorcio  \href{http://mvisionconsortium.mit.edu/}{\textit{M+Visión}}. El objetivo del proyecto fue el desarrollo de una herramienta para calcular automáticamente el ratio RV/LV sobre los ventrículos del corazón en imágenes de de angiografía por TC, biomarcador empleado en el pronóstico de embolia pulmonar.\vspace{0.2cm} \newline \textbf{- Logros} \newline Realicé una corta estancia en Boston en octubre del 2013 en el MIT y el BWH para conocer al equipo con el que se había estado colaborando en el proyecto \textit{RV/LV}, finalizar los algoritmos e instalar la herramienta en un sistema del BWH para su uso por parte de médicos. Mi trabajo contribuyó a la obtención de una patente internacional y una licencia de software registrada en el MIT. Conseguí la calificación de MH en la defensa de mi PFC, y recibí el premio como mejor PFC en el sector ``Ingeniería y Medicina'' en el año 2013 otorgado por el COIT-AEIT.}


\cventry{Sept.2012 Jul.2013}{Student Researcher}{Collaboration grant awarded by the Ministry of Education, Culture and Sport for students with a great performance}{BIT Research Group}{ETSIT-UPM}{}
\hfill \begin{minipage}{0.85\textwidth}\small 
\textbf{- Responsibilities} \newline Use of tools to visualize and analyze medical imaging such as Amide, ITKsnap and Amira. Segmentation of human and animals heart ventricles from CT and Magnetic Resonance (RM) images using level set techniques implemented in Matlab.  Application of detection algorithms based on machine learning techniques. \vspace{0.2cm}\newline \textbf{-Main project:} \textit{Right ventricular to left ventricular diameter ratio (RV/LV).} \newline In collaboration with UPM, \textit{(MIT)} and \textit{(BWH)} under the umbrella of \href{http://mvisionconsortium.mit.edu/}{\textit{M+Visión} consortium.} The goal of this project was the implementation of a system to compute automatically the RV/LV diameter ratio from CTPA images. This biomarker establishes a relation between the sizes of the right and left ventricles and can be employed in clinical practice to give a prognosis assessment to patients with pulmonary embolism. \vspace{0.2cm} \newline \textbf{- Achievement Highlights} \newline \begin{itemize} \item[$*$] My work has contributed to get an international patent, a software license registered in MIT and a few papers. \item[$*$] I was awarded by the "Colegio Oficial de Ingenieros de Telecomunicación" with the "Best Engineering Master Thesis of Spain 2013" in the health domain. Moreover I got Master Thesis with Honors. \item[$*$] In October 2013, I stayed in Boston for few weeks , with the objective of knowing in person the team from MIT and BWH in the context of \textit{RV/LV} project, finishing the algorithms and installing the tool in a system in BWH to be used by doctors. \end{itemize}
\end{minipage}


%% -- version 2
% Utilización de herramientas de visualización y análisis de imágenes médicas como Amide, ITKsnap y Amira. Segmentación de ventrículos de corazón humanos en imágenes de angiografía por TC y de cerdos en imágenes de resonancia magnética (RM) usando técnicas de level set implementadas en Matlab. Trabajo con algoritmos de detección basados en técnicas de aprendizaje máquina.  Tareas englobadas en el proyecto internacional \textit{“Right Ventricular to Left Ventricular Diameter Ratio”}, y que resultaron en el PFC. 

% --- antes

% Utilización de Matlab y herramientas de visualización de imágenes médicas como Amide, ITKsnap y Amira. Segmentación de ventrículos de corazón humanos y de cerdos en imágenes de CTPA y de RM respectivamente. Trabajo con algoritmo de detección basados en técnicas de machine-learning, modelos deformables y modelos mixtos. Participación en un proyecto internacional que dio lugar al PFC.


\section{Languages}
\cvlanguage{Spanish}{Native language.}{}
\cvlanguage{English}{Full professional competence.}{}%{Certificado TOEIC mayo 2012. Calificación B2}
\cvlanguage{French}{Basic competence.}{}

\section{Technical capabilities}

%\href{http://es.linkedin.com/in/sararodriguezlopez}{sararodriguezlopez} https://github.com/acil-bwh/ChestImagingPlatform

\begin{multicols}{2} 
\cvline{\href{http://www.cs.waikato.ac.nz/ml/weka/index.html}{\SARAMOUSE Weka}}{\centering \includegraphics[height=0.3cm]{./imagenes/4_5.png}\vspace{-.5cm}}
\cvline{\href{http://amide.sourceforge.net/}{\SARAMOUSE Amide}}{\centering \includegraphics[height=0.275cm]{./imagenes/5_5.png}\vspace{-.5cm}}
\cvline{\href{http://www.fei.com/software/amira-3d-for-life-sciences/}{\SARAMOUSE Amira}}{\centering \includegraphics[height=0.289cm]{./imagenes/3_5.png}\vspace{-.5cm}}
\cvline{\href{http://www.itksnap.org/pmwiki/pmwiki.php}{\SARAMOUSE ITKsnap}}{\centering \includegraphics[height=0.275cm]{./imagenes/5_5.png}\vspace{-.5cm}}
\cvline{\href{http://teem.sourceforge.net/}{\SARAMOUSE Teem}}{\centering \includegraphics[height=0.3cm]{./imagenes/4_5.png}\vspace{-.5cm}}
\cvline{\href{https://github.com/acil-bwh/ChestImagingPlatform}{\SARAMOUSE CIP}}{\centering \includegraphics[height=0.275cm]{./imagenes/5_5.png}\vspace{-.5cm}} 
\cvline{Matlab}{\centering \includegraphics[height=0.275cm]{./imagenes/5_5.png}\vspace{-.5cm}}
\cvline{Bash}{\centering \includegraphics[height=0.275cm]{./imagenes/5_5.png}\vspace{-.5cm}}
\cvline{Python}{\centering \includegraphics[height=0.289cm]{./imagenes/2_5.png}\vspace{-.5cm}}
\cvline{Java}{\centering \includegraphics[height=0.289cm]{./imagenes/2_5.png}\vspace{-.5cm}}
\cvline{mySQL}{\centering \includegraphics[height=0.289cm]{./imagenes/2_5.png}\vspace{-.5cm}}
\cvline{LaTeX}{\centering \includegraphics[height=0.275cm]{./imagenes/5_5.png}\vspace{-.5cm}}
\end{multicols} 


% \cvline{SO}{Linux (Ubuntu), Windows XP, Vista y Mac OS X.}
% \cvline{Ofimática}{MS Word, Excel, PowerPoint, Adobe Photoshop y LaTeX.}
% \cvline{Programación}{Python, Bash, Java, HTML, mySQL.}
% %\cvline{Diseño Gráfico}{Adobe Photoshop.}
% \cvline{Ingeniería}{Matlab, Weka.}
% \cvline{Imágenes médicas}{Amide, Amira, ITKsnap, Chest Imaging Platform, Librería Teem.}
% 
% % 









% \section{Principales proyectos de investigación}
% \cventry{Oct.2013-Actualidad}{Computer Aided Detection of Pulmonary Embolism (CADPE)}{UPM-MIT-BWH-Unidad Central de Radiodiagnóstico (UCR)}{}{}{El objetivo del proyecto ha sido el desarrollo y la optimización de un algoritmo de detección automática de tromboembolismos pulmonares sobre imágenes de CTPA. Utilizando el algoritmo presentado al Challenge CAD-PE del ISBI 2013 por el equipo “BWH-Baseline” como punto de partida, se ha desarrollado un algoritmo de aprendizaje básico basado en características locales de imagen. Ésta herramienta será liberada y entrará a formar parte de la librería de tratamiento de imágenes médicas Chest Imaging Platform, desarrollado por el Laboratorio Applied Chest Imaging Lab (ACIL) del BWH. }
% 
% \cventry{Oct.2012-Dic.2013}{Right ventricular to left ventricular diameter ratio (RV/LV)}{UPM-MIT-BWH}{}{}{El objetivo del proyecto fue el desarrollo de una herramienta para calcular automáticamente el ratio RV/LV sobre imágenes de CTPA. Este biomarcador relaciona el tamaño de los ventrículos del corazón y se emplea en el diagnóstico de embolia pulmonar. Se implantó dicha herramienta en un ordenador del BWH para su uso por parte de los radiólogos.}
% 
% %En colaboración con Germán González postPhd del MIT y el doctor Frank John Rybicki del BWH en el marco del programa M+Vision.


% \section{Proyecto final de carrera}
% \cventry{Julio 2013}{``Development of algorithms for automatic detection and segmentation of the heart in computerized tomography images''}{Desarrollado en el grupo de investigación BIT-ETSIT-UPM en colaboración con el MIT (Boston, EEUU) y el BWH (Boston, EEUU) en el marco del programa M+Vision}{}{}{En este proyecto se participó en el desarrollo de una herramienta automática para el cálculo del ratio RV/LV sobre imágenes de tomografía computerizada. Este biomarcador relaciona el tamaño de los ventrículos del corazón y se emplea en el diagnóstico de embolia pulmonar. \newline \textbf{Calificación: 10, Matrícula de Honor.}}
% 
% \section{Trabajo final de máster}
% \cventry{Julio 2014}{``Optimización de un sistema de detección automática de embolia pulmonar a partir de imágenes de CTPA''}{Desarrollado en el grupo de investigación BIT-ETSIT-UPM en colaboración con el MIT (Boston, EEUU), el BWH (Boston, EEUU) y la Unidad Central de Radiodiagnóstico (Madrid) en el marco del programa M+Vision}{}{}{El objetivo del TFM fue colaborar en el desarrollo y optimización de un algoritmo de detección automática de tromboembolismos pulmonares sobre imágenes de CTPA. Siendo el punto de partida el algoritmo presentado al Challenge CAD-PE del ISBI 2013 por el equipo “BWH-Baseline”, un algoritmo de aprendizaje básico basado en características locales de imagen. \newline \textbf{Calificación: 10, Matrícula de Honor.}}
\section{Patents and copyrights}

%\subsection{Patentes y licencias}

\cvline{$\bullet$}{Raul San Jose Estepar, Daniel Jimenez-Carretero, Sara Rodriguez-Lopez, Jorge Onieva Onieva, Maria Jesus Ledesma Carbayo, Frank J. Rybicki and German Gonzalez Serrano.  \textit{2015} \textbf{"Software to Determine the Prognosis of Patients Suffering from Pulmonary Embolism"}. Copyright. M.I.T. Case No. 17522J } 

\cvline{$\bullet$}{Daniel Jimenez-Carretero, Maria Jesus Ledesma Carbayo, Sara Rodriguez-Lopez, Frank J. Rybicki, Raul San Jose Estepar and German Gonzalez Serrano.  \textit{2013} \textbf{“Method to Determine the Prognosis of Patients Suffering from Pulmonary Embolism”}. Patente U.S. 61/909,574. Patente PCT P10393PC00. Provisional patent.}


\section{Honors and Awards}
% \cventry{Julio 2014}{Calificación de Matrícula de Honor en la defensa del trabajo final de máster}{"Optimización de un sistema de detección automática de embolia pulmonar a partir de imágenes de CTPA"}{}{}{}{}{}
\cventry{July 2014}{Biomedical Engineering Master's Final Project, \textbf{Grade: 10}}{``Optimización de un sistema de detección automática de embolia pulmonar a partir de imágenes de CTPA''}{}{}{}{}{}


\cventry{May 2014}{\href{https://www.coit.es/index.php?op=actos_premios_1567}{Winner of the ASISA price for the best Final Master Project in the Engineering and Medicine category of 2013, granted by the COIT+AEIT (Spanish Professional Organization of Telecommunication Engineers)}}{for "Development of Algorithms for Automatic Detection and Segmentation of the Heart in Computerized Tomography Images"}{\href{http://mvisionconsortium.mit.edu/news/team-ct-member-wins-best-masters-thesis-honor}{más información}}{}{}{}{}
% \cventry{Julio 2013}{Calificación de Matrícula de Honor en la defensa del proyecto final de carrera}{"Development of Algorithms for Automatic Detection and Segmentation of the Heart in Computerized Tomography Images"}{}{}{}{}{}
\cventry{July 2013}{Telecommunications Engineering Master's Final Project, \textbf{Grade: 10 with Honors}}{``Development of algorithms for automatic detection and segmentation of the heart in computerized tomography images''}{}{}{}{}{}

\section{Publications}

\cvline{$\bullet$}{Sara Rodríguez-López; Daniel Jimenez-Carretero; Raúl San José Estépar; Eduardo Fraile Moreno; Kanako K. Kumamaru; Frank J. Rybicki; Maria J. Ledesma-Carbayo; Germán González Serrano. \textbf{Automatic Ventricle Detection in Computed Tomography Pulmonary Angiography}. IEEE International Symposium on Biomedical Imaging: From Nano to Macro (ISBI 2015)}

\cvline{$\bullet$}{ German Gonzalez; Daniel Jimenez-Carretero; Sara Rodriguez-Lopez; Kanako K. Kumamaru; Elizabeth George; Raul San Jose Estepar; Frank J. Rybicky; Maria J. Ledesma-Carbayo. \textbf{Automated Axial Right Ventricle to Left Ventricle Diameter Ratio Computation in Computed Tomography Pulmonary Angiography}. PLOS ONE, 2015.}


%\cvline{$\bullet$}{Kanako K Kumamaru, Elizabeth George, Ayaz Aghayev, Sachin Saboo, Ashish Khandelwal, Sara Rodriguez-Lopez, Tianrun Cai,Daniel Jimenez-Carretero, Raúl San José Estépar, Maria J Ledesma-Carbayo, Germán González, Frank J Rybicki. \textbf{Implementation and performance of automated software to compute the RV/LV diameter ratio from CT pulmonary angiography images}. Journal of Thoracic Imaging. \textit{En revisión.}}

\cvline{$\bullet$}{German Gonzalez, Kanako K Kumamaru, Daniel Jimenez-Carretero, Elizabeth George, Maria J. Ledesma-Carbayo, Frank J Rybicki, Sara Rodriguez-Lopez, and Raul San Jose Estepar. \textbf{Automated Axial Right Ventricle to Left Ventricle Diameter Ratio Computation in Computed Tomography Pulmonary Angiography}. In RSNA Meeting, December 2013.}



% entender compra imbio y meter la licencia del código y tal. Preguntar a German

%METHOD AND SYSTEM FOR DETERMINING THE PROGNOSIS OF A PATIENT SUFFERING FROM PULMONARY EMBOLISM (USPTO Provisional patent application #: 61/909,574)

%aqui cuenta qu poner en el asunto patente http://www.agencia.mincyt.gob.ar/upload/guia_CV.pdf

% \subsection{Publicaciones}
% 
% \cvline{$\bullet$}{Sara Rodríguez-López; Daniel Jimenez-Carretero; Raúl San José Estépar; Eduardo Fraile Moreno; Kanako K. Kumamaru; Frank J. Rybicki; Maria J. Ledesma-Carbayo; Germán González Serrano. \textbf{Automatic Ventricle Detection in Computed Tomography Pulmonary Angiography}. IEEE International Symposium on Biomedical Imaging: From Nano to Macro (ISBI 2015), pp. XX-XXX. Brooklyn (NY, USA), Apr. 2015}
% 
% \cvline{$\bullet$}{ German Gonzalez; Daniel Jimenez-Carretero; Sara Rodriguez-Lopez; Kanako K. Kumamaru; Elizabeth George; Raul San Jose Estepar; Frank J. Rybicky; Maria J. Ledesma-Carbayo. \textbf{Automated Axial Right Ventricle to Left Ventricle Diameter Ratio Computation in Computed Tomography Pulmonary Angiography}. PLOS ONE XX 2015 XXX SUBMITED- preaprobado}
% 
% \cvline{$\bullet$}{German Gonzalez, Kanako K Kumamaru, Daniel Jimenez-Carretero, Elizabeth George, Maria J. Ledesma-Carbayo, Frank J Rybicki, Sara Rodriguez-Lopez, and Raul San Jose Estepar. \textbf{Automated Axial Right Ventricle to Left Ventricle Diameter Ratio Computation in Computed Tomography Pulmonary Angiography}. In RSNA Meeting, December 2013.}
% 


% \section{Formación complementaria}
% 
% \cvline{2011-2012}{Curso Práctico de Informática Sanitaria.}
% \cvline{2009-2012}{Sesiones sobre Tecnología, Desarrollo y Sociedad (jornadas Ongawa).}
% \cvline{2008-2009}{Conferencias sobre Tecnología para el Desarrollo Humano.}
% \cvline{2008-2009}{Acceso a la Información en Ingeniería y Arquitectura: aplicación práctica de
% los recursos de la Biblioteca Universitaria.}



% 
% \cvcomputer{\textbf{Operating Systems}}{
% \begin{itemize}
% \item Windows XP, Windows Vista
% \item Linux based: Ubuntu 10.10, etc.
% \end{itemize}}
% {\textbf{Development\\Languages}}{
% \begin{itemize}
% \item C++, C\#
% \item HTML, PHP, mySQL
% \item Java, Android
% \end{itemize}}
% \cvcomputer{\textbf{Matlab}}{
% \begin{itemize}
% \item Advanced programming and plotting
% \item Graphical User Interfaces
% \end{itemize}}
% {\textbf{Graphical Design}}{
% \begin{itemize}
% \item AutoCAD 2D-3D
% \item SolidEdge 3D
% \end{itemize}}
% \cvcomputer{Simulation Environments}{Webots}{}{}
% \cvcomputer{category 3}{XXX, YYY, ZZZ}{category 6}{XXX, YYY, ZZZ}
% 
% % 
% \section{Research Interests}
% \cvline{DATMO}{\small Description}
% \cvline{hobby 2}{\small Description}
% \cvline{hobby 3}{\small Description}
% 



% \cvlistitem{Item 2}
% \cvlistitem[+]{Item 3}            % optional other symbol
% 
% \section{Extra 2}
% \cvlistdoubleitem[\Neutral]{Item 1}{Item 4}
% \cvlistdoubleitem[\Neutral]{Item 2}{Item 5}
% \cvlistdoubleitem[\Neutral]{Item 3}{}

% % Publications from a BibTeX file
% \nocite{*}
% \bibliographystyle{plain}
% \bibliography{publications}       % 'publications' is the name of a BibTeX file


%\section{Publicaciones}
%\cvline{Junio 2013}{Solicitud a la oficina de patentes del MTI del estudio de la potencial patentabilidad y comercialización del método y el software desarrollado para el cáculo del biomarcador RV/LV en el marco del proyecto de investigación CADPE.}
%\cvline{Julio 2013}{Aceptación del abstract ``Automated Axial Right Ventricle to Left Ventricle Diameter Ratio Computation in Computed Tomography Pulmonary Angiography (CTPA)'' relacionado con el trabajo desarrollado en el proyecto CADPE, para la participación en el congreso de radiología RSNA en Chicago (EEUU) en diciembre del 2013.  }
% \section{Otros datos de interés}
% %\cvline{2009-2010}{Curso ``Cambio Climatico: Hechos, Implicaciones y Actuaciones''.}
% \cvline{2005-2009}{Clases particulares de matemáticas, física y dibujo técnico a nivel ESO y Bachillerato.}
% \cvline{2004-2007}{Miembro activo en el grupo de teatro Valdecás}
% \cvline{2008 y 2009}{Realización de dos cursos anuales de “Improvisación Teatral” por el grupo de teatro \textit{No Es Culpa Nuestra}}
% \cvline{2009 y 2010}{Realización de cursos de improvisación e interpretación teatral por \textit{Replika Teatro} y \textit{Teatro Asura}.}
% \cventry{2010-2011}{Participante en el proyecto Destaca}{}{Desarrollo de Estudiantes de Alto Potencial}{}{Este proyecto, organizado por la UPM, ha elegido entre sus alumnos de ETSI Aeronáutica, ETSI Telecomunicación, ETSI Topografía Geodesia y Cartografía, EUIT Aeronáutica, EU Informática y EU Arquitectura Técnica,  a aquellos que poseen más alto rendimiento para evaluar su potencial. El objetivo del proyecto es desarrollar las capacidades de estos alumnos al máximo para incrementar sus posibilidades de éxito a la hora de adaptarse adecuadamente a las demandas de las empresas.}
% \cvline{Desde 2011}{Participación en cursos de bailes latinos}
\end{document}


%% end of file `template_en.tex'.(Networked Multi-Robot Systems)
