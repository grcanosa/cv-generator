%CV in SPA
%% start of file `template.tex'.
%% Copyright 2006-2013 Xavier Danaux (xdanaux@gmail.com).
%
% This work may be distributed and/or modified under the
% conditions of the LaTeX Project Public License version 1.3c,
% available at http://www.latex-project.org/lppl/.



\documentclass[11pt,a4paper,sans]{moderncv}        % possible options include font size ('10pt', '11pt' and '12pt'), paper size ('a4paper', 'letterpaper', 'a5paper', 'legalpaper', 'executivepaper' and 'landscape') and font family ('sans' and 'roman')

% moderncv themes
\moderncvstyle{classic}                             % style options are 'casual' (default), 'classic', 'oldstyle' and 'banking'
\moderncvcolor{red}                               % color options 'blue' (default), 'orange', 'green', 'red', 'purple', 'grey' and 'black'
%\renewcommand{\familydefault}{\sfdefault}         % to set the default font; use '\sfdefault' for the default sans serif font, '\rmdefault' for the default roman one, or any tex font name
%\nopagenumbers{}                                  % uncomment to suppress automatic page numbering for CVs longer than one page

% character encoding
\usepackage[utf8]{inputenc}                       % if you are not using xelatex ou lualatex, replace by the encoding you are using
%\usepackage{CJKutf8}                              % if you need to use CJK to typeset your resume in Chinese, Japanese or Korean

% adjust the page margins
\usepackage[top    = 2.75cm,
bottom = 2.50cm,
left   = 2.50cm,
right  = 3.00cm]{geometry}
\setlength{\hintscolumnwidth}{2cm}
\AtBeginDocument{\recomputelengths}
%\setlength{\hintscolumnwidth}{3cm}                % if you want to change the width of the column with the dates
\setlength{\makecvtitlenamewidth}{10cm}           % for the 'classic' style, if you want to force the width allocated to your name and avoid line breaks. be careful though, the length is normally calculated to avoid any overlap with your personal info; use this at your own typographical risks...





% personal data
\firstname{\Huge Rodr\'iguez Canosa,}
\familyname{\\ \huge Gonzalo Ruy}
%\name{Gonzalo}{Rodr\'iguez Canosa}
\title{Doctor in Robótica}               % optional, remove the line if not wanted
\extrainfo{Fecha de nacimiento: 07-12-1984} 
%\address{Madrid (SPAIN)}{}    % optional, remove the line if not wanted
\address{C/Agustín Durán 30, 1ºIZQ}{28028 Madrid (SPAIN)}    % optional, remove the line if not wanted
\phone[mobile]{+34 617599017}                    % optional, remove the line if not wanted
%\phone{+34 913363010}                      % optional, remove the line if not wanted
%\fax{fax (optional)}                          % optional, remove the line if not wanted
\email{grcanosa@gmail.com}                      % optional, remove the line if not wanted
\social[linkedin]{grcanosa} 
%\homepage{http://es.linkedin.com/in/grcanosa/en}
\photo[64pt][0pt]{gonzalo.jpg}  





% % personal data
% \name{John}{Doe}
% \title{Resumé title}                               % optional, remove / comment the line if not wanted
% \address{street and number}{postcode city}{country}% optional, remove / comment the line if not wanted; the "postcode city" and "country" arguments can be omitted or provided empty
% \phone[mobile]{+1~(234)~567~890}                   % optional, remove / comment the line if not wanted; the optional "type" of the phone can be "mobile" (default), "fixed" or "fax"
% \phone[fixed]{+2~(345)~678~901}
% \phone[fax]{+3~(456)~789~012}
% \email{john@doe.org}                               % optional, remove / comment the line if not wanted
% \homepage{www.johndoe.com}                         % optional, remove / comment the line if not wanted
% \social[linkedin]{john.doe}                        % optional, remove / comment the line if not wanted
% \social[twitter]{jdoe}                             % optional, remove / comment the line if not wanted
% \social[github]{jdoe}                              % optional, remove / comment the line if not wanted
% \extrainfo{additional information}                 % optional, remove / comment the line if not wanted
% \photo[64pt][0.4pt]{picture}                       % optional, remove / comment the line if not wanted; '64pt' is the height the picture must be resized to, 0.4pt is the thickness of the frame around it (put it to 0pt for no frame) and 'picture' is the name of the picture file
% \quote{Some quote}                                 % optional, remove / comment the line if not wanted

% to show numerical labels in the bibliography (default is to show no labels); only useful if you make citations in your resume
%\makeatletter
%\renewcommand*{\bibliographyitemlabel}{\@biblabel{\arabic{enumiv}}}
%\makeatother
%\renewcommand*{\bibliographyitemlabel}{[\arabic{enumiv}]}% CONSIDER REPLACING THE ABOVE BY THIS

% bibliography with mutiple entries
%\usepackage{multibib}
%\newcites{book,misc}{{Books},{Others}}
%----------------------------------------------------------------------------------
%            content
%----------------------------------------------------------------------------------

\definecolor{color0}{rgb}{0,0,0}% black
\definecolor{color1}{rgb}{0.545098,0,0}% burgundy
%\definecolor{color1}{rgb}{0.2,0.2,0.2}% Testcolor
\definecolor{color2}{rgb}{0.35,0.35,0.35}% dark grey

\quote{Doctor Ingeniero con amplia experiencia en desarrollo de proyectos hw+sw. 5 años de experiencia demostrables.}   

\begin{document}
%\begin{CJK*}{UTF8}{gbsn}                          % to typeset your resume in Chinese using CJK
%-----       resume       ---------------------------------------------------------
\makecvtitle

\section{Experiencia profesional}
\GONCVsecPROFESIONALEXPERIENCEeng
\subsection{Experiencia laboral}
\cventry{Feb 2014 \\presente}{Analista Programador Senior}{eProsima}{Madrid}{Spain}{Implementación del protocolo RTPS (Real Time Publish-Subscribe) definido por la OMG, centrandose en rendimiento y seguridad}
\cventry{Feb 2009 \\Nov 2013}{Investigador predoctoral}{Grupo de Robótica y Cibernética}{Madrid}{Spain}{Sistemas Multirobot para la vigilancia y seguridad de Infrastructuras Críticas.}
\cventry{Sep 2011 \\Dec 2011}{Investigador visitante}{Grupo de robótica (UoA)}{Auckland}{New Zealand}{Sistema de detección de objetos móviles desde UAV basado en flujos ópticos, bajo la supervisión de Dr. Bruce MacDonald.}
\cventry{May 2008 \\Ago 2008}{Ingeniero en prácticas}{Dpto. Mecánica de vuelo, Bombardier}{Montreal}{Canada}{Interfaz gráfica para la optimización del proceso de documentación de los informes de simulación y pruebas de vuelo desarrollada en Matlab.}
              % arguments 3 to 6 are optional
\subsection{Becas}
\cventry{Feb 2008 \\Abr 2008}{Beca de investigación}{Dpto. de Ingeniería de Transportes}{Universidad de Sevilla}{}{Estudio de dispositivos CANBUS de automóviles, analizando datos y protocolos de distintas marcas.}
\cventry{Nov 2007 \\Ene 2008}{Proyecto de Fin de Carrera}{Dpto. de Ingeniería de Sistemas y Automática}{Universidad de Sevilla}{}{Interfaz para la definición de problemas de transporte y almacenaje; así como un sistema de control centralizado para la obtención de la solución óptima.}

\section{Formación}
PHD
MSC
ERASMUS
US
Instituto
Colegio



\section{Idiomas}
XX
\cvlanguage{Inglés}{Competencia profesional completa}{Nivel alto hablado y escrito.}
\cvlanguage{German}{Competencia profesional básica.}{Nivel medio hablado y escrito.`Zentrale MittelstuffePrüfung` del Instituto Goethe (C1).}

\newpage
% \section{Computer skills}
% \cvline{OS}{Windows XP, Vista, Linux (Ubuntu).}
% \cvline{Office}{MS Word, Excel, PowerPoint and LaTeX.}
% \cvline{Programming}{C, C++, Matlab, HTML.}
% \cvline{Simulation}{Webots, Player-Gazebo, Simulink.}
% \cvline{Robotics}{Robot Operating System (ROS), 3D Vision, OpenCV, PCL (Point Cloud Library), OpenGL.}
% %\cvline{Engeeniering}{Advanced Matlab Programming, SolidEdge 3D, AutoCAD 2D-3D.}

\section{XX}
\subsection{XX}
\cvline{SO}{Linux and Windows.}
\cvline{Office Suite}{MS Word, Excel, PowerPoint, LaTeX.}
\cvline{C++}{Large experience in different projects with multiple libraries (Boost, OpenCV, ...). Experience with performance improvements techniques and profilers tools.}
\cvline{Programming}{Beginner level of C\#, Python, Django, HTML, Java, Android SDK, OpenGL.}
\cvline{Development}{CMake, Git, SVN, Eclipse, Microsoft Visual Studio, ...}
\subsection{XX}
\cvline{Simulation}{Webots, Player-Gazebo, Simulink.}
\cvline{Matlab}{Fast prototyping of algorithms. Graphical Interfaces.}
\cvline{Robotics}{Mobile and aerial robotics (UGV y UAV), Robot Operating System (ROS).}
\cvline{Vision}{Expert development in OpenCV and PCL (Point Cloud Library).}
\cvline{Hardware}{Experience in sensor and actuator control. Data adquisition platforms and mechanisms design. Experience with LIDAR sensors and 3D cameras.}


\section{ XX}
\cvline{2011 - 2013}{ROTOS national project. Multirobot Systems for protection of large critical infraestructures. Ref: DPI2010-17998 of DPI subprogram}
\cvline{2009 - 2010}{Networked Multi-Robot Systems (NMRS) project by the European Defense Agency. B-0004-ESM2\_ERG.}

 \section{XX}
 \cvline{$\bullet$}{ G. Rodríguez-Canosa, A. Barrientos y J. Cerro \textit{Detection and Tracking of Dynamic
Objects by Using a Multirobot System: Application to Critical Infrastructures Surveillance} Sensors. vol.14 -2 pags. 2911 - 2943. 2014.}
 \cvline{$\bullet$}{D. Sanz, G. Rodríguez-Canosa, J. Barrientos, J. Cerro, J. Hernandez, A. Barrientos. \textit{Sensorized robotic sphere for large exterior critical infrastructures supervision} Journal of Applied Remote Sensing vol.7 pags. 1-19 (F.I. .876) 2013. JCR: 1.953}
\cvline{$\bullet$}{Gonzalo R. Rodríguez-Canosa, Stephen Thomas, Jaime del Cerro, Antonio Barrientos and Bruce MacDonald.\textit{A Real-Time Method to Detect and Track Moving Objects (DATMO) from Unmanned Aerial Vehicles (UAVs) Using a Single Camera}. Remote Sensensing. 2012, 4(4), 1090-1111; doi:10.3390/rs4041090. JCR: 2.171}
% \cvline{$\bullet$}{Gonzalo R. Rodríguez-Canosa, J.Del Cerro and A. Barrientos. \textit{Detección e Identificación de Objetos Móviles en Sistemas Multi-Robot con Información 3D}. 8th Workshop of Robocity 2030-II on Exterior Robots. Madrid, November 2010.}

\section{XX}
\cvline{Title}{Detection and Tracking of Dynamic Objects. A MultiRobot Approach to Critical Infrastructures Surveillance.}
\cvline{Supervisors}{Jaime del Cerro and Antonio Barrientos.}
\cvline{Description}{My thesis was  focused in MultiRobot Systems for security and surveillance of Exterior Critical Infrastructures (CI). I have developed detection of dynamic objects algorithms for both aerial and ground robots. A Point Cloud based robot localization system has also been developed. Main skills and tools used in this thesis include but are not limited to: Mobile robotics, Simulation, Webots, ROS, PCL, OpenCV. \mbox{Download: \href{https://www.dropbox.com/s/tn7ooep4quv1nvl/PhD.GonzaloRCanosa.HR.4oct.pdf}{\textit{\textcolor{color1}{thesis.pdf}}}}}


\section{Extra Curricular Activities}
\cvline{July 2012}{Cooperation of Robots and Sensor Networks. GKmM Summer School 2012.\newline \url{http://www.gkmm.tu-darmstadt.de/summerschool/?q=node/1}}
\cvline{2011}{Organization of \textit{Cybertech 2011} contest. The contest objective is to build and program a robot able to follow a line; passing by other vehicles and avoiding obstacles; and also find the exit through a maze. Organizing the contest involves teaching three courses to the participants and supervise the developments of some of the groups, providing support.}
\cvline{Sept. 2010}{Participation in CEABOT 2010 contest. Humanoid robots contest with three main tests: a) walking avoiding obstacles, b) stairs climbing, and c) sumo fights with other robots. 4th place out of 11 groups from Spanish universities.}
\cvline{Nov. 2010}{Participation and oral presentation in 8th Workshop of Robocity 2030-II on Exterior Robots in Automatic and Robotic Center (CAR) in Madrid.}
\cvline{2007}{Signals, Sensors and Systems Seminar in KTH, Stockholm.}

\section{Awards and Memberships}
\cvline{2011}{Student Member of IEEE}
\cvline{2010}{Student Member of CEA-IFAC - Comité Español de Automática (Spanish Automation Committee).}
\cvline{2010}{Student Member of Robocity 2030-II Robotics Consortium.}
\cvline{2002}{Secondary Education with Honors by Junta de Andalucía, Spain.}


% 
% \section{Education}
% \cventry{year--year}{Degree}{Institution}{City}{\textit{Grade}}{Description}  % arguments 3 to 6 can be left empty
% \cventry{year--year}{Degree}{Institution}{City}{\textit{Grade}}{Description}
% 
% \section{Master thesis}
% \cvitem{title}{\emph{Title}}
% \cvitem{supervisors}{Supervisors}
% \cvitem{description}{Short thesis abstract}
% 
% \section{Experience}
% \subsection{Vocational}
% \cventry{year--year}{Job title}{Employer}{City}{}{General description no longer than 1--2 lines.\newline{}%
% Detailed achievements:%
% \begin{itemize}%
% \item Achievement 1;
% \item Achievement 2, with sub-achievements:
%   \begin{itemize}%
%   \item Sub-achievement (a);
%   \item Sub-achievement (b), with sub-sub-achievements (don't do this!);
%     \begin{itemize}
%     \item Sub-sub-achievement i;
%     \item Sub-sub-achievement ii;
%     \item Sub-sub-achievement iii;
%     \end{itemize}
%   \item Sub-achievement (c);
%   \end{itemize}
% \item Achievement 3.
% \end{itemize}}
% \cventry{year--year}{Job title}{Employer}{City}{}{Description line 1\newline{}Description line 2}
% \subsection{Miscellaneous}
% \cventry{year--year}{Job title}{Employer}{City}{}{Description}
% 
% \section{Languages}
% \cvitemwithcomment{Language 1}{Skill level}{Comment}
% \cvitemwithcomment{Language 2}{Skill level}{Comment}
% \cvitemwithcomment{Language 3}{Skill level}{Comment}
% 
% \section{Computer skills}
% \cvdoubleitem{category 1}{XXX, YYY, ZZZ}{category 4}{XXX, YYY, ZZZ}
% \cvdoubleitem{category 2}{XXX, YYY, ZZZ}{category 5}{XXX, YYY, ZZZ}
% \cvdoubleitem{category 3}{XXX, YYY, ZZZ}{category 6}{XXX, YYY, ZZZ}
% 
% \section{Interests}
% \cvitem{hobby 1}{Description}
% \cvitem{hobby 2}{Description}
% \cvitem{hobby 3}{Description}
% 
% \section{Extra 1}
% \cvlistitem{Item 1}
% \cvlistitem{Item 2}
% \cvlistitem{Item 3. This item is particularly long and therefore normally spans over several lines. Did you notice the indentation when the line wraps?}
% 
% \section{Extra 2}
% \cvlistdoubleitem{Item 1}{Item 4}
% \cvlistdoubleitem{Item 2}{Item 5\cite{book1}}
% \cvlistdoubleitem{Item 3}{Item 6. Like item 3 in the single column list before, this item is particularly long to wrap over several lines.}
% 
% \section{References}
% \begin{cvcolumns}
%   \cvcolumn{Category 1}{\begin{itemize}\item Person 1\item Person 2\item Person 3\end{itemize}}
%   \cvcolumn{Category 2}{Amongst others:\begin{itemize}\item Person 1, and\item Person 2\end{itemize}(more upon request)}
%   \cvcolumn[0.5]{All the rest \& some more}{\textit{That} person, and \textbf{those} also (all available upon request).}
% \end{cvcolumns}
% 
% % Publications from a BibTeX file without multibib
% %  for numerical labels: \renewcommand{\bibliographyitemlabel}{\@biblabel{\arabic{enumiv}}}% CONSIDER MERGING WITH PREAMBLE PART
% %  to redefine the heading string ("Publications"): \renewcommand{\refname}{Articles}
% \nocite{*}
% \bibliographystyle{plain}
% \bibliography{publications}                        % 'publications' is the name of a BibTeX file

% Publications from a BibTeX file using the multibib package
%\section{Publications}
%\nocitebook{book1,book2}
%\bibliographystylebook{plain}
%\bibliographybook{publications}                   % 'publications' is the name of a BibTeX file
%\nocitemisc{misc1,misc2,misc3}
%\bibliographystylemisc{plain}
%\bibliographymisc{publications}                   % 'publications' is the name of a BibTeX file

% \clearpage
% %-----       letter       ---------------------------------------------------------
% % recipient data
% \recipient{Company Recruitment team}{Company, Inc.\\123 somestreet\\some city}
% \date{January 01, 1984}
% \opening{Dear Sir or Madam,}
% \closing{Yours faithfully,}
% \enclosure[Attached]{curriculum vit\ae{}}          % use an optional argument to use a string other than "Enclosure", or redefine \enclname
% \makelettertitle
% 
% Lorem ipsum dolor sit amet, consectetur adipiscing elit. Duis ullamcorper neque sit amet lectus facilisis sed luctus nisl iaculis. Vivamus at neque arcu, sed tempor quam. Curabitur pharetra tincidunt tincidunt. Morbi volutpat feugiat mauris, quis tempor neque vehicula volutpat. Duis tristique justo vel massa fermentum accumsan. Mauris ante elit, feugiat vestibulum tempor eget, eleifend ac ipsum. Donec scelerisque lobortis ipsum eu vestibulum. Pellentesque vel massa at felis accumsan rhoncus.
% 
% Suspendisse commodo, massa eu congue tincidunt, elit mauris pellentesque orci, cursus tempor odio nisl euismod augue. Aliquam adipiscing nibh ut odio sodales et pulvinar tortor laoreet. Mauris a accumsan ligula. Class aptent taciti sociosqu ad litora torquent per conubia nostra, per inceptos himenaeos. Suspendisse vulputate sem vehicula ipsum varius nec tempus dui dapibus. Phasellus et est urna, ut auctor erat. Sed tincidunt odio id odio aliquam mattis. Donec sapien nulla, feugiat eget adipiscing sit amet, lacinia ut dolor. Phasellus tincidunt, leo a fringilla consectetur, felis diam aliquam urna, vitae aliquet lectus orci nec velit. Vivamus dapibus varius blandit.
% 
% Duis sit amet magna ante, at sodales diam. Aenean consectetur porta risus et sagittis. Ut interdum, enim varius pellentesque tincidunt, magna libero sodales tortor, ut fermentum nunc metus a ante. Vivamus odio leo, tincidunt eu luctus ut, sollicitudin sit amet metus. Nunc sed orci lectus. Ut sodales magna sed velit volutpat sit amet pulvinar diam venenatis.
% 
% Albert Einstein discovered that $e=mc^2$ in 1905.
% 
% \[ e=\lim_{n \to \infty} \left(1+\frac{1}{n}\right)^n \]
% 
% \makeletterclosing

%\clearpage\end{CJK*}                              % if you are typesetting your resume in Chinese using CJK; the \clearpage is required for fancyhdr to work correctly with CJK, though it kills the page numbering by making \lastpage undefined
\end{document}


%% end of file `template.tex'.
